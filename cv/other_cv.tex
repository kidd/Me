\documentclass{resume} % Use the custom resume.cls style

\usepackage[left=0.75in,top=0.6in,right=0.75in,bottom=0.6in]{geometry} % Document margins
\usepackage{hyperref}

\name{Raimon Grau} % Your name
\address{ \url{raimonster@gmail.com} \\ \url{http://puntoblogspot.blogspot.com} \\ \url{https://github.com/kidd/Me} }

\address{@Barcelona OR @Tenerife \\ Spain} % Your address
%% \address{(+34)~$\cdot$~696809713 \\ raimonster@gmail.com} % Your phone number and email

\begin{document}

%----------------------------------------------------------------------------------------
%	EDUCATION SECTION
%----------------------------------------------------------------------------------------

\begin{rSection}{Education}

{\bf Facultat Informatica de Barcelona (UPC), Barcelona} \hfill {\em January 2010} \\
B.S. in Computer Science \\
Final thesis on Evolutionary Algorithms graded with Honours. \smallskip \\
\end{rSection}

%----------------------------------------------------------------------------------------
%	WORK EXPERIENCE SECTION
%----------------------------------------------------------------------------------------

\begin{rSection}{Experience}
  \begin{rSubsection}{Kong Inc - \url{https://konghq.com}}{February 2018 - }{Software Engineer, Tech Lead}{Remote}
  \item Designing features for the enterprise grade customers of
    Kong. Usually higher traffic and custom flows that do not apply to
    the opensource product. LuaJIT
  \item Design and implement a data collection tool from request
    logs. Sanitizing and anonymising data, triggering jobs for
    processing that data for several purposes (ML, logs, docs,
    service discovery). Python, Flask, PG, Celery, Docker.
  \item Design and implement collector for generating an OpenAPI spec
    from real traffic seen from live request/reponse
    pairs. Organizations have running systems they don't know how they
    work or what they use. Imagine a way to draw the spec from real
    traffic, with 0 configuration. Python.
  \item Redesign branching and merging strategy for better integration
    of teams companywise. Go from a manual cherrypicking model to a
    consolidated strategy where every developer is able to merge and
    produce releases of products.
  \item Managing Releases for Enterprise Kong product. Increasing the
    cadence of releases and smoothing the CI/CD pipeline to make easier
    for newcommers to cut releases, debug packages, etc. Jenkins,
    docker, and Bash. Lots of Bash.
  %% \item Mentoring developers to increase code quality and
  %%   testablility.
  %% \item Aligning Engineering and Product agree on feasible plans of
  %%   execution.
  %% \item Help designing the interviewing/hiring process.
  \item Relevant Books: Turn the Ship Around, Release It!, Peopleware,
    The Mythical Man-Month, Deep Work, The Art of Postgresql.
  \end{rSubsection}


%------------------------------------------------

\begin{rSubsection}{RavenPack International - \url{https://ravenpack.com}}{June 2016  - August 2017}{Software Engineer, Common Lisp Developer}{Marbella, Spain}
\item Modernizing a 10 y.o Common Lisp codebase, adding a test suite
  and a build system for it migrating it into continuous integration.
\item Refactoring of an old application for performance and
  maintainability. Halved the number of lines with better performance
  while improving correctness along the way.
\item Writting a streamming api for advanced filtering of information
  (think twitter firehose with more advanced filters). Using Common Lisp
  for backend and Nginx+Nchan for the distribution.  Delivering 2.5M
  unique events per day.
\item Improving a REST API performance by using worker queues when
  possible in different parts of the system. Using both internal
  in-memory queues and AWS SQS. Common Lisp + Python.
\item Relevant Books: SICP, Object Oriented Reengineering Patterns,
  The Unix Programming Environment, Patterns of Artificial
  Intelligence Programming, The Information (Gleick), On Lisp.
\end{rSubsection}

%------------------------------------------------

\begin{rSubsection}{3scale, Inc - \url{http://www.3scale.net/}}{July 2011 - June 2016}{Software Engineer, Team Lead}{Barcelona, Spain}
\item Maintaining and Developing a +200k lines Ruby on Rails
  application to manage API authentication and rate limitings. Team
  lead since 2014.
\item Develop a code generator to create Nginx Lua configurations from
  rules introduced by the users. Also run nginx on cloud to host these
  configurations. Developed in Ruby, Lua, Nginx and Redis.
\item Work in an Nginx+Lua solution to cache calls to external apis to
  reduce the outbound traffic and be able to not get hit by rate
  limits (500rps). Highly concurrent calls and heavy redis usage. Lua,
  Nginx and Redis.
\item Trie based router for api request routing. cost:
  O(current\_route.length) instead of O(total\_routes.length).
%% \item Write an openresty module (lua) to perform the full oauth2
%%   authentication process (both 2 and 3 legged).
%% \item Work in apitools.com (http://www.apitools.com) project from the
%%   beginning until current state. Nginx platform + docker to allow
%%   users to proxy all the outbound traffic through apitools and monitor
%%   and code middleware to modify the API calls. Did a talk on that in
%%   ApiDays Barcelona 2014. Basic usage of Docker.
%% \item 3scale itself is an API management company, so I've been dealing
%%   with different kinds of API models, authentications,
%%   rate\_limitings, documentation models, testing, etc...
\item Relevant Books: SICP, Ruby Metaprogramming, GodelEscherBach,
  Refactoring.
\end{rSubsection}

%------------------------------------------------

%% \begin{rSubsection}{Gnuine}{December 2010 - July 2011}{Web Developer}{Barcelona, Spain}
%% \item First contact with web development, Ruby, and Rails. Working in
%%   different Ruby on Rails projects. First serious testing and Git
%%   usage.
%% \end{rSubsection}

%------------------------------------------------

%% \begin{rSubsection}{Intelligent Pharma - http://intelligentpharma.com/}{July 2008 - December 2010}{Developer}{Barcelona, Spain}
%% \item Developing Artificial intelligence software for drug discovery.
%% \item Evolutionary algorithms and Genetic programming.
%% \item Development in Perl, Java, C++ and using unix everywhere.
%% \item coarse grained paralelization of jobs running in supercomputers
%%   (MareNostrum, the 2nd biggest supercomputer at that time)
%% \item Higher Order Perl, Git Internals, Perl*, Effective Java, Java
%%   Concurrency in Practice.
%% \end{rSubsection}


%% \begin{rSubsection}{AMES S.A.}{September 2007 - 2003}{Developer}{Barcelona, Spain}
%% \item Developing internal applications in Dataflex, Crystal Reports, and Perl.
%% \item Developing industrial apps to manage machinery by reading barcodes.
%% \end{rSubsection}

\end{rSection}

%----------------------------------------------------------------------------------------
%	TECHNICAL STRENGTHS SECTION
%----------------------------------------------------------------------------------------

\begin{rSection}{Skills}

\item Programming Languages enthusiast. Working experience with Lua,
  CommonLisp, Ruby, Perl, Python, Bash, a bit of js \& ts. Hobbist
  experience with Smalltalk, Elisp, Clojure, Elixir, Rust, Go, Nim...

\item Working experience with REST, Postgres, Docker, sqlite , Redis,
  MongoDB, Redshift, Oracle, Cassandra, development\&productivity
  tools like deep knowledge of Shell, Tmux , Vim, Emacs, git,
  Openresty (nginx+lua), Nix. Experience on elasticsearch, Redis,
  Statsd + Graphite + Grafana, AWS (experience with dynamodb, ec2, s3,
  cloudwatch).

\item Passionate learner, and willing to transmit enthusiasm for
  learning technologies and balance them with the current stable
  stack. I'm quite cautious on ``The next big thing'' hypes (there is
  no silver bullet), and like to usually keep things very simple
  (\url{http://suckless.org}). I believe simplicity and uniformity
  sometimes wins 'the perfect tool for the job' x number-of-jobs
  approach.

\item Good communication skills and enthusiast teacher \& learner.

\item In my past life I used to be into reverse engineering
  software. Most of it is outdated now, but my approach to problem
  solving stayed.

\item I have a wide variety of interests from Live Programming (Lisp,
  Smalltalk), machine learning, compilers, blockchain technologies...

\item I love attending small conferences. And try to attend at least 1
  conference per year as a healthy industry catchup measure.

\item Opensource contributor. Contributions to several projects like
  emacs, ripgrep, Nix, and several Emacs modes.
\end{rSection}


%% \begin{rSection}{Projects and Free Software}

%%   I'm a convinced advocate of Free software, and I have several
%%   projects opensourced in github (http://www.github.com/kidd), and
%%   I've contributed to several others. Also I've been involved in
%%   several open source communities. For example:

%% \item Member of Barcelona Perl Mongers.
%% \item Member of Smalltalk.cat.
%% \item Patch published in Vimperator (2008)
%% \item Closely following Emacs development. Committed some code. There's
%%   much to learn from a 30 year old software project at both code level
%%   and management level.
%% \item (ex-)member of Vectorlinux crew. I developed a few apps to
%%   package apps for vectorlinux and maintaining the repos. Also a
%%   migration tool for old packages to new packages, tracking
%%   dependencies and recursively building them in preorder.
%% \item Following GNU Guix development and IRC.  I package and maintain
%%   a few packages (luajit, openresty, xmlstarlet).
%% \item \href{https://github.com/areina/helm-dash}{Helm-dash.el}
%%   (coauthor/maintainer). Browse documentation offline from within
%%   emacs. It reads Dash docsets. (~300 stars)
%% \item
%%   \href{https://github.com/kidd/erc-image.el}{erc-image.el} /
%%   \href{https://github.com/kidd/erc-youtube.el}{erc-youtube.el}
%%   (author/maintainer). Visualize images inline in erc irc
%%   client. (Included in spacemacs project)
%% \item \href{https://github.com/apitools/router.lua}{router.lua}
%%   (author/maintainer). Trie-based router in lua using unification
%%   instead of simple matching, so that traveling a route is
%%   O(route.size).
%% \item \href{https://github.com/kidd/eva}{eva.lua}. supersimple and
%%   toy-like lisp interpreter in lua
%% \item \href{https://github.com/kidd/hascheme}{hascheme}(half assed
%%   scheme). not so supersimple but still toy-like scheme interpreter in
%%   Perl.
%% \item \href{https://github.com/kidd/redditfs.lua}{redditfs.lua}. fuse
%%   module written in lua to mount reddit as a file system.
%% \item More on github. I've written are some pet projects I wrote in
%%   elixir/factor/elisp/Io \ldots
%% \item Backend of a geocaching-like app in clojure using ring,
%%   compojure, secretary, korma, postgres+GIS and google cloud
%%   messaging. Frontend of the app for mobile, using cordova and Google
%%   maps api.
%% \end{rSection}

\begin{rSection}{Personal}

\item I love music (from rock and jazz to funk and psy \ldots)
\item Reading is another passion of mine. Check out what I'm up to at
  \url{https://www.goodreads.com/user/show/69378962-raimon-grau}.
\item I used to play Guitar and bass guitar. Self taught, not
  practicing much anymore...
%% \item I love good conversations over beers about programming
%%   languages
\item I blog in \url{http://puntoblogspot.blogspot.com} since
  2008. Sharing mostly technical stuff. Lately I moved some of the
  writting to \url{https://raimonster.com}
\item I currently live in
  \href{https://www.google.es/maps/place/Tenerife/@28.2925426,-17.0803948,9z/data=!3m1!4b1!4m2!3m1!1s0xc4029effe8682ed:0xb01a4bf1c84baf3c}{Tenerife}.
\item A complementary CV is on \href{https://github.com/kidd/Me}{https://github.com/kidd/Me}
\end{rSection}

%----------------------------------------------------------------------------------------
%	EXAMPLE SECTION
%----------------------------------------------------------------------------------------

%\begin{rSection}{Section Name}

%Section content\ldots

%\end{rSection}

%----------------------------------------------------------------------------------------

\end{document}
