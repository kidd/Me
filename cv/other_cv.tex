\documentclass{resume} % Use the custom resume.cls style

\usepackage[left=0.75in,top=0.6in,right=0.75in,bottom=0.6in]{geometry} % Document margins
\usepackage{hyperref}

\name{Raimon Grau} % Your name
\address{ \url{raimonster@gmail.com} \\ \url{http://puntoblogspot.blogspot.com} \\ \url{https://github.com/kidd/Me} }

\address{@Barcelona OR @Tenerife \\ Spain} % Your address
%% \address{(+34)~$\cdot$~696809713 \\ raimonster@gmail.com} % Your phone number and email

\begin{document}

%----------------------------------------------------------------------------------------
%	EDUCATION SECTION
%----------------------------------------------------------------------------------------

\begin{rSection}{Education}

{\bf Facultat Informatica de Barcelona (UPC), Barcelona} \hfill {\em January 2010} \\
B.S. in Computer Science \\
Final thesis on Evolutionary Algorithms graded with Honours. \smallskip \\
\end{rSection}

\begin{rSection}{Languages}
Spanish - native\\
Catalan - native\\
English - high (written and spoken)\\
\end{rSection}
%----------------------------------------------------------------------------------------
%	WORK EXPERIENCE SECTION
%----------------------------------------------------------------------------------------

\begin{rSection}{Experience}
  \begin{rSubsection}{Kong Inc - \url{https://konghq.com}}{February 2018 - }{Software Engineer, Tech Lead}{Remote}
  \item Contributing to features for the enterprise version of Kong
    product. Producing plugins and enterprise features like workspaces
    & RBAC. Lua
  \item Design a data collection tool based on a Kong
    plugin. Offloading traffic data from Kong nodes to an API
    endpoint. Collecting data, sanitizing and anonymising, and
    triggering jobs for processing that data for several purposes (ML,
    logs, docs, discovery). Python+Flask+PG+Celery+Docker.
  \item Generating a Swagger and OpenAPI specs from real traffic seen
    from live request/reponse pairs. Python
  \item Managing Releases for Enterprise Kong product. Increasing the
    peace of releases and smoothing the CI/CD pipeline to make easier
    for newcommers to cut releases, debug packages, etc. Jenkins,
    docker, and Bash. Lots of bash.
\end{rSubsection}

%------------------------------------------------

\begin{rSubsection}{RavenPack International - \url{https://ravenpack.com}}{June 2016  - August 2017}{Software Engineer, Common Lisp Developer}{Marbella, Spain}
\item Modernizing a 10 y.o Common Lisp codebase adding a test suite
  and a build system for it migrating it into continuous
  integration.
\item Writting a streamming api for advanced filtering of information
  (think twitter hose with more advanced filters). Using Common Lisp
  for backend and Nginx+Nchan for the distribution.  Delivering 2.5M
  unique events per day.
\item Improving a REST API performance by using worker queues when
  possible in different parts of the system. Using both internal
  in-memory queues and AWS SQS. Common Lisp + Python.
\end{rSubsection}

%------------------------------------------------

\begin{rSubsection}{3scale, Inc - \url{http://www.3scale.net/}}{July 2011 - June 2016}{Software Engineer, Team Lead}{Barcelona, Spain}
\item Maintaining and developing a Ruby on Rails application to manage
  API authentication and rate limitings. Team lead since 2014.
\item Develop a code generator to create Nginx Lua configurations from
  rules introduced by the users. Also run nginx on cloud to host these
  configurations. Developed in Ruby, Lua, Nginx and Redis.
\item Work in an Nginx+Lua solution to cache calls to external apis to
  reduce the outbound traffic and be able to not get hit by rate
  limits (500rps). Highly concurrent calls and heavy redis usage. Lua,
  Nginx and Redis.
\item Write an openresty module (lua) to perform the full oauth2
  authentication process (both 2 and 3 legged).
\item Work in apitools.com (http://www.apitools.com) project from the
  beginning until current state. Nginx platform + docker to allow
  users to proxy all the outbound traffic through apitools and monitor
  and code middleware to modify the API calls. Did a talk on that in
  ApiDays Barcelona 2014. Basic usage of Docker.
\item 3scale itself is an API management company, so I've been dealing
  with different kinds of API models, authentications,
  rate\_limitings, documentation models, testing, etc...
\end{rSubsection}

%------------------------------------------------

%% \begin{rSubsection}{Gnuine}{December 2010 - July 2011}{Web Developer}{Barcelona, Spain}
%% \item First contact with web development, Ruby, and Rails. Working in
%%   different Ruby on Rails projects. First serious testing and Git
%%   usage.
%% \end{rSubsection}

%------------------------------------------------

\begin{rSubsection}{Intelligent Pharma - http://intelligentpharma.com/}{July 2008 - December 2010}{Developer}{Barcelona, Spain}
\item Developing Artificial intelligence software for drug discovery.
\item Evolutionary algorithms and Genetic programming.
\item Development in Perl, C++ and using unix everywhere.
\item coarse grained paralelization of jobs running in supercomputers
  (MareNostrum, the 2nd biggest supercomputer at that time)
\end{rSubsection}


%% \begin{rSubsection}{AMES S.A.}{September 2007 - 2003}{Developer}{Barcelona, Spain}
%% \item Developing internal applications in Dataflex, Crystal Reports, and Perl.
%% \item Developing industrial apps to manage machinery by reading barcodes.
%% \end{rSubsection}

\end{rSection}

%----------------------------------------------------------------------------------------
%	TECHNICAL STRENGTHS SECTION
%----------------------------------------------------------------------------------------

\begin{rSection}{Skills}

\item Computer Languages: Ruby, Perl, Lua, Common Lisp, Elisp,
  Clojure, Smalltalk.  I'm a programming language fanatic so I've
  dabbled with lots of others (either at university or for fun):
  C, Rust, Python, Prolog, C++, Java, Scheme, Assembly, Dylan, Io, Factor,
  Forth, Shen, Erlang, Elixir, Haskell, Oz, and written some very
  basic Lisp interpreters (see github).

\item Protocols \& APIs: Working experience with HTTP, JSON, XML,
  REST, SQL, Redis, MongoDB, Redshift, Oracle, Sqlite. Interest in and
  side projects with lots of others: EventStore, Tarantool, influxdb,
  Gemstone/S, DataFlex, cocroachdb.

\item Development tools: Deep knowledge of zsh, screen, Vim, Emacs
  (VERY heavy user for 5 years, written some plugins and contributed
  code to emacs itself), git, Openresty (nginx+lua). Knowledge of
  elasticsearch, redis, statsd+Graphite+grafana, AWS (experience with
  dynamodb, ec2, s3, cloudwatch).

\item Passionate learner, and willing to transmit enthusiasm for
  learning technologies and balance them with the current stable
  stack. I'm quite cautious on ``The next big thing'' hypes (there is
  no silver bullet), and like to usually keep things very simple
  (\url{http://suckless.org}). I believe simplicity and uniformity
  sometimes wins 'the perfect tool for the job' x number-of-jobs.

\item Good communication skills and enthusiast teacher \& learner.

\item I used to be into reverse engineering software (1997-2000). Most
  of it is outdated now, but my approach to problem solving stayed.

\item I have a wide variety of interests from reverse engineering to
  machine learning, compilers, blockchain technologies, etc.

\end{rSection}

\begin{rSection}{Conferences}

  I love attending conferences (preferably not very crowded/mainstream
  ones). I'm trying to attend at least 1 conference per year as a
  healthy measure (it's good for the soul).

%% \item LambdaWorld 2017. (Cadiz, Spain)
%% \item J on the Beach 2017. (Malaga, Spain)
%% \item LambdaWorld 2016. (Cadiz, Spain)
%% \item Euroclojure 2015. (Barcelona, Spain)
%% \item Polyconf 2014. (Postdam, Poland)
%% \item ApiDays Barcelona 2014. As a Speaker
%% \item ApiStrat EUrope 2014. As staff. (Amsterdam)
%% \item Lua Workshop 2103 (Toulouse, France)
%% \item European Lisp Symposium 2013 (Madrid, Spain)
%% \item Fosdem 2013 (Brussels, Belgium)
%% \item YAPC:EU (perl) 2009. (Lisbon, Portugal)
%% \item PRACE Supercomputing 2009 (Barcelona, Spain)

\end{rSection}

\begin{rSection}{Projects and Free Software}

  I'm a convinced advocate of Free software, and I have several
  projects opensourced in github (http://www.github.com/kidd), and
  I've contributed to several others. Also I've been involved in
  several open source communities. For example:

\item Member of Barcelona Perl Mongers.
\item Member of Smalltalk.cat.
\item Patch published in Vimperator (2008)
\item Closely following Emacs development. Committed some code. There's
  much to learn from a 30 year old software project at both code level
  and management level.
\item (ex-)member of Vectorlinux crew. I developed a few apps to
  package apps for vectorlinux and maintaining the repos. Also a
  migration tool for old packages to new packages, tracking
  dependencies and recursively building them in preorder.
\item Following GNU Guix development and IRC.  I package and maintain
  a few packages (luajit, openresty, xmlstarlet).
\item \href{https://github.com/areina/helm-dash}{Helm-dash.el}
  (coauthor/maintainer). Browse documentation offline from within
  emacs. It reads Dash docsets. (~300 stars)
\item
  \href{https://github.com/kidd/erc-image.el}{erc-image.el} /
  \href{https://github.com/kidd/erc-youtube.el}{erc-youtube.el}
  (author/maintainer). Visualize images inline in erc irc
  client. (Included in spacemacs project)
\item \href{https://github.com/apitools/router.lua}{router.lua}
  (author/maintainer). Trie-based router in lua using unification
  instead of simple matching, so that traveling a route is
  O(route.size).
\item \href{https://github.com/kidd/eva}{eva.lua}. supersimple and
  toy-like lisp interpreter in lua
\item \href{https://github.com/kidd/hascheme}{hascheme}(half assed
  scheme). not so supersimple but still toy-like scheme interpreter in
  Perl.
\item \href{https://github.com/kidd/redditfs.lua}{redditfs.lua}. fuse
  module written in lua to mount reddit as a file system.
\item More on github. I've written are some pet projects I wrote in
  elixir/factor/elisp/Io \ldots
\item Backend of a geocaching-like app in clojure using ring,
  compojure, secretary, korma, postgres+GIS and google cloud
  messaging. Frontend of the app for mobile, using cordova and Google
  maps api.
\end{rSection}

\begin{rSection}{Personal}

\item I love music (from rock and jazz to funk and psy \ldots).
\item Reading is another passion of mine. (``Obviously'', my favourite
  technical books are those that are known by their initials:
  \begin{enumerate}
  \item \href{https://mitpress.mit.edu/sicp/}{SICP}
  \item \href{https://en.wikipedia.org/wiki/G%C3%B6del,_Escher,_Bach}{GEB}
  \item \href{http://norvig.com/paip.html}{PAIP}
  \item \href{http://hop.perl.plover.com/}{HOP}
  \item \href{https://www.info.ucl.ac.be/~pvr/book.html}{CTM}
  \end{enumerate}
\item I used to play Guitar and bass guitar. (very bad)
\item I also love good conversations over beers about programming
  languages.

\item I blog in \url{http://puntoblogspot.blogspot.com} since
  2008. Sharing mostly technical stuff.
\item I like wordplays.
\item I currently live in \href{https://www.google.es/maps/place/Tenerife/@28.2925426,-17.0803948,9z/data=!3m1!4b1!4m2!3m1!1s0xc4029effe8682ed:0xb01a4bf1c84baf3c}{Tenerife}.
\item My Cv is on \href{https://github.com/kidd/Me}{Github}
\end{rSection}

%----------------------------------------------------------------------------------------
%	EXAMPLE SECTION
%----------------------------------------------------------------------------------------

%\begin{rSection}{Section Name}

%Section content\ldots

%\end{rSection}

%----------------------------------------------------------------------------------------

\end{document}
